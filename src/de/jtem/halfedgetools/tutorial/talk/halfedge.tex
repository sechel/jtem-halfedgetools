% This document is the template for the talks given at the Matheon Center Days 30. M�rz bis 1. April 2009.
%
%Please time your presentation for 10 minutes plus 5 minutes questioning.
%
% The beamerMatheon class was created for the
%
%          DFG Research Center Matheon
%          Mathematics for Key Technologies
%
% Please send corrections and suggestions to webmaster@matheon.de
%
% use the beamer class:
\documentclass[12pt]{beamer}

% use the beamerMatheon theme with english titles
\mode<presentation>{\usetheme[language=german]{Matheon}}

% used packages
\usepackage[T1]{fontenc}
\usepackage[utf8]{inputenc}
\usepackage{amsmath}
\usepackage{amssymb}
\usepackage{alltt}
\usepackage{url}
\beamertemplatenavigationsymbolsempty
\usepackage[3D]{movie15}
\usepackage{overpic}
\usepackage{contour} \contourlength{0.2ex}
\usepackage{array}

\def\lput(#1,#2)#3{\put(#1,#2){\hbox to 0pt{\hss{#3}}}}
\def\cput(#1,#2)#3{\put(#1,#2){\hbox to 0pt{\hss{#3}\hss}}}

\def\xw{{\mathbf x}}


% ----------- title ------------------------------------------

\title{\vspace{-1cm}Glas, Stahl und Geometrie}
\subtitle{Mathematik in der modernen Architektur}
\author{Stefan Sechelmann}
\date
\insertLogoTU
% \footinformation{Short title of project}

% ----------- document ------------------------------------------
\begin{document}
\maketitle


%\begin{frame}
%\frametitle{\"Ubersicht}
%\tableofcontents
%\end{frame}


% ----------- frames ------------------2-------------------------

\section{Einleitung}

%  Architekten haben Ideen unabhängig von der Ausführung
\begin{frame}
\frametitle{Worum geht es?}
\begin{center}
\includegraphics[height=7.5cm]{bilder/sketchesoffrankgehry_scene_14.jpg}\\	
\end{center}
\tiny{Tanzendes Hause, Frank Gehry, Prag 1996}
\end{frame}

\begin{frame}
\frametitle{Worum geht es?}
\begin{center}
\includegraphics[height=7.5cm]{bilder/501305399_efd871f2ec_b.jpg} \\
\end{center}
\tiny{Tanzendes Hause, Frank Gehry, Prag 1996}
\end{frame}


% Eine Glasfassade kan nicht aus einer Glasscheibe gebaut werden
% Desshalb Unterteilung in einzelne Scheiben
\begin{frame}
\frametitle{Glassfassade ist nicht aus einem St\"uck}
\begin{center}
\includegraphics[height=7.5cm]{bilder/Prague_-_Dancing_House.jpg}
\end{center}
\tiny{Tanzendes Hause, Frank Gehry, Prag 1996}
\end{frame}


\section{Dreiecke}
\subsection{Beispiele}

\begin{frame}
\frametitle{Viele M\"oglichkeiten zum Beispiel Dreiecke}
\includegraphics[height=8cm]{bilder/BMW_Welt01.jpg}
\end{frame}

\begin{frame}
\frametitle{BMW Welt von Innen}
\begin{center}
\includegraphics[height=8cm]{bilder/42153_zoomimage.jpg}
\end{center}
\end{frame}

\begin{frame}
\frametitle{Zlote Tarasy}
\begin{center}
\includegraphics[height=8cm]{bilder/Zlote_Tarasy01.jpg}
\end{center}
\end{frame}

\begin{frame}
\frametitle{Zlote Tarasy}
\begin{center}
\includegraphics[width=12cm]{bilder/Zlote_Tarasy_W2.jpg}
\end{center}
\end{frame}

\begin{frame}
\frametitle{Zlote Tarasy}
\begin{center}
\includegraphics[height=7cm]{bilder/Zlote_Tarasy_Construct01.jpg} \\
\end{center}
Dreiecke wie in Computerspielen
\end{frame}

\subsection{In Computerspielen}

\begin{frame}
\frametitle{Dreiecksgitter in Computerspielen}
\begin{center}
\includegraphics[width=12cm]{bilder/Mesh02.png}
\end{center}
\end{frame}

\section{Vierecke}
\subsection{Gekr\"ummtes Glas}

\begin{frame}
\frametitle{Oder Vierecke}
\begin{center}
\includegraphics[height=8cm]{bilder/173698278_8fdf66ac79_b.jpg}
\end{center}
\end{frame}

\begin{frame}
\frametitle{Von Oben}
\begin{center}
\includegraphics[height=8cm]{bilder/zepp-cam_10.jpg}
\end{center}
\end{frame}

\begin{frame}
\frametitle{U-Bahn Ausgang in Paris}
\begin{center}
\includegraphics[width=10cm]{bilder/sortie.jpg}
\end{center}
\tiny{U-Bahn Ausgang Saint Lazare Paris RFR Ingenieure}
\end{frame}

\begin{frame}
\frametitle{Gebogenes Glas meistens ist (zu) teuer}
\begin{center}
\includegraphics[height=8cm]{bilder/Nordkettenbahnen_Station01.jpg}
\end{center}
\end{frame}

\subsection{Ebenes Glas}

\begin{frame}
\frametitle{Preiswerter - Ebene Glasscheiben}
\includegraphics[width=12cm]{bilder/berlin-hauptbahnhof_01_gross.jpg}
\end{frame}




\begin{frame}
\frametitle{Mathematische Optimierung}
\begin{center}
\begin{tabular}{rl}
$8 \%$ &
\includegraphics[height=4cm]{bilder/vorher.png}\\
$0 \%$&
\includegraphics[height=4cm]{bilder/nachher2.png}\\
\end{tabular}
\end{center}
\end{frame}

\begin{frame}
\frametitle{Verschiedene Definitionen von Ebenheit}
\begin{center}
\scalebox{0.6} {
\input{unebenheit.pdf_t}
}
\end{center}
\end{frame}


\begin{frame}
\frametitle{Demo Planarizer}
\begin{center}
\includegraphics[width=11cm]{bilder/Planarizer.png}
\end{center}
\end{frame}

\begin{frame}
\frametitle{Energie, Funktional}
\begin{eqnarray*}
E(M)&=&\sum_{f_i\in M}e(f_i)\\
&=&e(f_1)+e(f_2)+\dots+e(f_n)
\end{eqnarray*}

\begin{tabular}{ll}
	$M$ & - gesamte Oberfl\"ache \\
	$f_i$ & - $i$-te Glasscheibe \\
	$e(f_i)$ & - Energie/Unebenheit der $i$-ten Glasscheibe \\
	$E(M)$ & - Energie/Unebenheit der gesamten Oberfl\"ache
\end{tabular}

\end{frame}

\begin{frame}
\frametitle{Optimierung}
\begin{eqnarray*}
E(M)&=&\sum_{f_i\in M}e(f_i)\\
&=&e(f_1)+e(f_2)+\dots+e(f_n)
\end{eqnarray*}

\begin{alertblock}{Ziel}
\begin{eqnarray*}
E(M)&=&klein \\
E'(M)&=&0
\end{eqnarray*}
\end{alertblock}
\end{frame}


\begin{frame}
\frametitle{Newtonverfahren}
\begin{center}
\includegraphics[width=9cm]{bilder/Newton_iteration.png}
\end{center}
\end{frame}


\begin{frame}
	\frametitle{The Opus, Zaha Hadid Architekten 2011}
	\begin{center}
	\includegraphics[height=5.5cm]{bilder/opus01.jpg}
	\includegraphics[height=5.5cm]{bilder/opus02.png}
	\end{center}
\end{frame}

\begin{frame}
	\begin{center}
	\frametitle{The Opus, Zaha Hadid Architekten 2011}
	\includegraphics[width=12cm]{bilder/opus03.png}
	\end{center}
\end{frame}


\begin{frame}
\frametitle{Ohne Optimierung}
\begin{center}
\includegraphics[height=7cm]{bilder/x1.jpg}
\end{center}
\tiny{Esplanade Singapur (Foto: Prof. Peter Schr\"oder)}
\end{frame}


\subsection{Einfach gekr\"ummtes Glas}

\begin{frame}
	\frametitle{Einfach gekr\"ummte Glasscheiben}
	\begin{center}
		\includegraphics[width=12cm]{bilder/tgv01.png}
	\end{center}
	\tiny{TGV Bahnhof Strasbourg, (c) RFR}
\end{frame}

\begin{frame}
	\frametitle{Einfach gekr\"ummte Glasscheiben}
	\begin{center}
		\includegraphics[height=8cm]{bilder/tgv02.png}
	\end{center}
\end{frame}


\begin{frame}
	\frametitle{Grenzwert bei Verfeinerung in eine Richtung}
	\scalebox{0.9} {
	\begin{picture}(.01,.01)
	\put(5,-55){
	\begin{overpic}[width=1.05\textwidth]{bilder/semidisc}
	\lput(0,2){{$\xw_{11}$}}
	\lput(3,11){{$\xw_{12}$}}
	\lput(3,20){{$\xw_{13}$}}
	\cput(9.5,-1){{$\xw_{21}$}}
	\lput(75,8){$\xw_1(u)$}
	\cput(83,12.5){\contour{white}{$\xw_2(u)$}}
	\boldmath
	\cput(30.5,15){$\to$}
	\cput(67,15){$\to\cdots\to$}
	\end{overpic}
	}
	\end{picture}
	}
%\tiny{Bild: Prof. Johannes Wallner, TU Graz}
\end{frame}


\subsection{Sechsecke}


\begin{frame}
\frametitle{Sechseckige Glasscheiben (Forschung)}
\begin{center}
\includegraphics[width=12cm]{bilder/hexa01.png}
\end{center}
\tiny{Bild: Prof. Helmut Pottman, Tu-Wien}
\end{frame}




\section{Muster auf Fassaden}

\subsection{Beispiele}

\begin{frame}
	\frametitle{Muster auf Fassaden}
	\includegraphics[width=12cm]{bilder/024-03bhamtraffic1.jpg}
\end{frame}

\begin{frame}
	\frametitle{Muster auf Fassaden}
	\begin{center}
	\includegraphics[height=8cm]{bilder/23-08birmingham6.jpg}
	\end{center}
\end{frame}

\subsection{Parametrisierung}

\begin{frame}
\frametitle{Beispiel Nationalbibliothek Prag}
	\begin{center}
	\includegraphics[height=8cm]{bilder/fs1.jpg}
	\end{center}
\end{frame}

\begin{frame}
\frametitle{Demo Parametrisierung}
	\begin{center}
	\includegraphics[width=11cm]{bilder/library02.png}
	\end{center}
	\tiny{Skalierung!}
\end{frame}



\begin{frame}
	\frametitle{Parametrisierung}
	\begin{center}
	\begin{tabular}{m{5.5cm}m{0.5cm}m{5.5cm}}
	$\Bbb R^2$ & & $\Bbb R^3$ \\
	\includegraphics[width=5.5cm]{bilder/param01.png} & $\to$ & \includegraphics[width=5.5cm]{bilder/library01.png} \\
	\end{tabular}
	\end{center}
\end{frame}


\section{Ende}

\begin{frame}
\frametitle{Ende}
\begin{center}
\includegraphics[height=4cm]{bilder/sketchesoffrankgehry_scene_14.jpg}
%\quad
%\includegraphics[height=4cm]{bilder/501305399_efd871f2ec_b.jpg}
\end{center}
Vielen Dank!
\end{frame}








% \begin{frame}
% \frametitle{Pyramiden, Gizeh $2500$ v.Chr}
% \includegraphics[width=12cm]{bilder/All_Gizah_Pyramids-2.jpg}
% \end{frame}
% 
% \begin{frame}
% \frametitle{Kolosseum, Rom $80$ n.Chr}
% \includegraphics[width=12cm]{bilder/Colosseum_in_Rome,_Italy_-_April_2007.jpg}
% \end{frame}

% \begin{frame}
% \begin{center}
% \frametitle{Sears Tower, Chicago 1974}
% \includegraphics[height=9cm]{bilder/KM_6167_sears_tower_august_2007_D.jpg}
% \end{center}
% \end{frame}
% 
% \begin{frame}
% \begin{center}
% \frametitle{Grande Arche, Paris 1989}
% \includegraphics[width=11cm]{bilder/Grande-Arche.jpg}
% \end{center}
% \end{frame}
% 
% \begin{frame}
% \frametitle{The Gherkin, London 2004}
% \begin{center}
% \includegraphics[width=11cm]{bilder/London01.jpg}
% \end{center}
% \end{frame}
% 
% 
% \begin{frame}
% \frametitle{Opus, Zaha Hadid Architects, Dubai 2011}
% \includegraphics[height=7cm]{bilder/opus-rendering-02.jpg}
% \includegraphics[height=6cm]{bilder/opussubdivision01.jpg}
% \end{frame}
% 
% \begin{frame}
% \frametitle{Tanzendes Haus, Prag 1996}
% \begin{center}
% \includegraphics[height=9cm]{bilder/Prague_-_Dancing_House.jpg}
% \end{center}
% \end{frame}
% 
% 
% \begin{frame}
% \frametitle{Freiform-Architektur}
% \begin{block}{Aufgabe}
% 	Modellierung einer Glasfassade durch ungekrümmte Glasscheiben
% \end{block}
% \begin{center}
% \includegraphics[width=12cm]{bilder/opussubdivision.jpg}
% \end{center}
% \end{frame}
% 
% \begin{frame}
% \frametitle{Optimierung der Glasscheiben}
% \begin{center}
% \includemovie[
% 	mimetype=model/u3d,
% 	label=Vorher,
% 	3Djscript=Encompass.js,
% 	autoplay=true
% ]
% { 12cm }
% { 4cm }
% {models/bendflat02.u3d}
% \includemovie[
% 	mimetype=model/u3d,
% 	label=Nachher,
% 	3Djscript=Encompass.js,
% 	autoplay=true
% ]
% { 12cm }
% { 4cm }
% {models/bendflat01.u3d}
% \end{center}
% \end{frame}


\end{document}
